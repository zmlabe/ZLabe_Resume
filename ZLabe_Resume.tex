%%%%%%%%%%%%%%%%%%%%%%%%%%%%%%%%%
%%% CV Jakša Tomović 06/2019 %%%%%%
%%%%%%%%%%%%%%%%%%%%%%%%%%%%%%%%%

\PassOptionsToPackage{dvipsnames}{xcolor}
\documentclass[10pt,a4paper]{altacv}

%Layout
\geometry{left=1cm,right=9cm,marginparwidth=6.8cm,marginparsep=1.2cm,top=1.45cm,bottom=1.25cm,footskip=2\baselineskip}

%Packages
\usepackage[utf8]{inputenc}
\usepackage[T1]{fontenc}
\usepackage[default]{lato}
\usepackage{hyperref}

%%% Added special
\usepackage{academicons}

%Colors

\definecolor{accent}{HTML}{000e17}
\definecolor{heading}{HTML}{000e17}
\definecolor{emphasis}{HTML}{696969}
\definecolor{body}{HTML}{000e17}

\colorlet{heading}{heading}
\colorlet{accent}{accent}
\colorlet{emphasis}{emphasis}
\colorlet{body}{body}

\renewcommand{\itemmarker}{{\small\textbullet}}
\renewcommand{\ratingmarker}{\faCircle}

%

\begin{document}
\name{Zachary M. Labe, Ph.D.}
\tagline{I am a trained atmospheric scientist trying to visualize the signal from a lot of noise.}
\personalinfo{
    \email{zachary.labe@noaa.gov}
    \phone{+1.609.452.6571}
    \location{Princeton, New Jersey 08540 USA}
    \homepage{zacklabe.com}
    \linkedin{linkedin.com/in/zacharylabe}
    \twitter{x.com/ZLabe}
    \github{github.com/zmlabe}
    \slides{slideshare.net/ZacharyLabe}
}

%

\begin{fullwidth}
\makecvheader
\end{fullwidth}

%

%%%%%%%%%%%%%%%%%%%%%%%%%%%%%%% Experience %%%%%%%%%%%%%%%%%%%%%%%%%%%%%%%
\vspace{-0.09in}
\cvsection[page1sidebar]{Background}
\begin{itemize}
    \setlength{\itemindent}{0.5em}
    \item[--] \small{Developed \& led innovative research on climate impacts \& machine learning}
    \item[--] \small{Published \textbf{33 peer-reviewed} scientific articles (journals/technical reports)}
    \item[--] \small{Collaborated with local\slash federal stakeholders and educational nonprofits}
    \item[--] \small{Presented more than \textbf{75 talks} for both technical \& non-specialist audiences}
    \item[--] \small{Conducted over \textbf{100 interviews} with local-to-international news media}
    \item[--] \small{Visualize \& communicate climate data on social media (\textbf{100,000+ followers})}
    \item[--] \small{Coordinated \textbf{6 sessions} at local workshops \& international climate meetings}
    \item[--] \small{Participated on \textbf{3} grant proposal panels \& reviewed over \textbf{100} journal studies}
    \item[--] \small{Highly experienced in working on large, interdisciplinary teams \& mentoring}
    \item[--] \small{Contributor to international global climate \& weather assessments annually}
    \item[--] \small{Honored as a Kavli Fellow of the National Academy of Sciences in 2019}
\end{itemize}
\smallskip

%%%%%%%%%%%%%%%%%%%%%%%%%%%%%%% Research and Work Experience %%%%%%%%%%%%%%%%%%%%%%%%%%%%%%%
\vspace*{-0.37cm}
\cvsection{Research \& Work Experience}

\cvevent{Research Physical Scientist (Federal)}{NOAA Geophysical Fluid Dynamics Laboratory (GFDL)}{June 2024 -- Ongoing}{Princeton, NJ}
\begin{itemize}
    \setlength{\itemindent}{0.5em}
    \item[--]   \small{Applying AI\slash ML methods to assess \& develop high-resolution climate models for improving climate prediction, projection, and risk assessment}
\end{itemize}
\medskip

%\cvevent{Associate Research Scholar}{Princeton University \& NOAA GFDL}{May 2024 -- June 2024}{Princeton, NJ}
%\begin{itemize}
%    \setlength{\itemindent}{0.5em}
%    \item[--]   \small{Applying explainable machine learning methods to output from Earth System Models for improving climate prediction and projection}
%\end{itemize}
%\medskip

\cvevent{Postdoc to Associate Research Scholar}{Princeton University \& NOAA GFDL}{May 2022 -- June 2024}{Princeton, NJ}
\begin{itemize}
    \setlength{\itemindent}{0.5em}
    \item[--]   \small{Designed a framework to attribute extreme events in near real-time using observations, climate models and other data-driven statistical methods}
\end{itemize}
\medskip

\cvevent{Postdoc}{Colorado State University}{June 2020 -- April 2022}{Fort Collins, CO}
\begin{itemize}
    \setlength{\itemindent}{0.5em}
    \item[--]   \small{Leveraged new explainable machine learning methods for extracting patterns of anthropogenic climate change from natural variability}
    \item[--]    \small{Awarded a Sustainability Leadership Fellowship at Colorado State University with formal training in science communication, policy, and outreach}
\end{itemize}
\medskip

\cvevent{Graduate Research Assistant}{University of California, Irvine}{September 2015 -- June 2020}{Irvine, CA}
\begin{itemize}
    \setlength{\itemindent}{0.5em}
    \item[--]   \small{Implemented new modeling experiments to address Arctic climate extremes}
    \item[--]    \small{Awarded National Science Foundation NRT Fellowship for data science}
\end{itemize}

\smallskip
\vspace*{0.03cm}

%%%%%%%%%%%%%%%%%%%%%%%%%%%%%%% Interests %%%%%%%%%%%%%%%%%%%%%%%%%%%%%%%
\vspace*{-0.37cm}
\cvsection{Interests}
\hspace*{-1.1cm} 
\wheelchart{1.1cm}{0.5cm}{%
20/12em/OliveGreen!40/High Resolution\\ Climate Models,
10/8em/OliveGreen!30/Climate Risk \& Impact Services,
30/9em/OliveGreen/Climate Change \& Extreme Weather,
5/13em/OliveGreen!10/Data Visualization \& Science Communication,
29/10em/OliveGreen!70/Explainable AI\slash ML \& Deep Learning,
6/10em/OliveGreen!30/Subseasonal-to-Decadal Climate Prediction
}


%%New Page
% \clearpage




\end{document}